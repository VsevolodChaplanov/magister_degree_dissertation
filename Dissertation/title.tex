% Титульный лист (ГОСТ Р 7.0.11-2001, 5.1)
\thispagestyle{empty}%
\begin{center}%
\MakeUppercase{
\textbf{МИНИСТЕРСТВО НАУКИ И ВЫСШЕГО ОБРАЗОВАНИЯ РОССИЙСКОЙ ФЕДЕРАЦИИ
Федеральное государственное автономное образовательное учреждение высшего образования «КАЗАНСКИЙ (ПРИВОЛЖСКИЙ) ФЕДЕРАЛЬНЫЙ УНИВЕРСИТЕТ»
Институт математики и механики им. Н.И. Лобачевского
Кафедра аэрогидромеханики
}}
\end{center}%
%
\vspace{0pt plus4fill} %число перед fill = кратность относительно некоторого расстояния fill, кусками которого заполнены пустые места

Направление подготовки (специальность): 01.04.03 - Механика и математическое моделирование

Профиль (специализация, магистерская программа): Механика жидкости газа и плазмы

\vspace{0pt plus4fill}
\begin{center}%
    \MakeUppercase{Магистерская диссертация}
\end{center}%

\vspace{0pt plus4fill}
\begin{center}%
    \MakeUppercase{Генерация турбулентного поля скорости для входных граничных условий}
\end{center}%

\begin{flushright}%
На правах рукописи

\textsl {УДК \thesisUdk}
\end{flushright}%
%
\vspace{0pt plus6fill} %число перед fill = кратность относительно некоторого расстояния fill, кусками которого заполнены пустые места
\begin{center}%
{\large \thesisAuthor}
\end{center}%
%
\vspace{0pt plus1fill} %число перед fill = кратность относительно некоторого расстояния fill, кусками которого заполнены пустые места
\begin{center}%
\textbf {\large \thesisTitle}

\vspace{0pt plus2fill} %число перед fill = кратность относительно некоторого расстояния fill, кусками которого заполнены пустые места
{%\small
Специальность \thesisSpecialtyNumber~---

<<\thesisSpecialtyTitle>>
}

\vspace{0pt plus2fill} %число перед fill = кратность относительно некоторого расстояния fill, кусками которого заполнены пустые места
Диссертация на соискание учёной степени

\thesisDegree
\end{center}%
%
\vspace{0pt plus4fill} %число перед fill = кратность относительно некоторого расстояния fill, кусками которого заполнены пустые места
\begin{flushright}%
Научный руководитель:

\supervisorRegalia

\supervisorFio
\end{flushright}%
%
\vspace{0pt plus4fill} %число перед fill = кратность относительно некоторого расстояния fill, кусками которого заполнены пустые места
\begin{center}%
{\thesisCity~--- \thesisYear}
\end{center}%
\newpage
