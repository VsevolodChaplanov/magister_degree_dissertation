%% Согласно ГОСТ Р 7.0.11-2011:
%% 5.3.3 В заключении диссертации излагают итоги выполненного исследования, рекомендации, перспективы дальнейшей разработки темы.
%% 9.2.3 В заключении автореферата диссертации излагают итоги данного исследования, рекомендации и перспективы дальнейшей разработки темы.

В настоящей работе были рассмотрены два подхода к генерации синтетической однородной изотропной турбулентности по
заданной функции энергетического спектра: спектральное разложение и стохастическое моделирование.

В рамках первого подхода был разработан собственный алгоритм генерации.
Переход к использованию одной гармоники Фурье позволил существенно сократить вычислительные затраты на генерацию одной флуктуации.
Явное выражение амплитуд мод Фурье с использованием энергетического спектра позволил провести более точную аппроксимацию спектра,
что в свою очередь дало более реалистичное поле сгенерированных турбулентных флуктуаций.

Для второго подхода были использованы известные общие методы
прямого и последовательного Гаусова моделирования.
Для метода прямого моделирования Гаусса была сделана попытка сокращения вычислительной сложности задачи за счёт
ограничения количества вычисляемых собственных чисел.
Метод последовательных симуляций ранее не использовался для генерации турбулентных флуктуаций, поэтому особое внимание было уделено валидации генерируемых полей.
На основе лучшего отражение целевого спектра
в сравнении с модифицированным спектральным методом мы показали, что предлагаемый подход является конкуретным для подобного рода задач.
В сравнении с прямым моделированием последовательные симуляции позволили не только в разы сократить требуемое для генерации время,
но также позволили рассматривать более подробные сетки и генерировать на них турбулентные флуктуации.

Сравнение методов было произведено на основе предложенного тестового кусочно-линейного спектра.
Суммируя полученные результаты, кратко выделим преимущества и недостатки рассматриваемых методов.
\newline
Достоинства и недостатки спектрального метода:
\begin{itemize}[label=+]
    \item Высокая скорость генерации одной реализации, в среднем 44 мс на 30'000 точек без ипользования параллельных вычислений;
    \item Устойчивость к топологии сетки, нет зависимости от соседник узлов;
    \item Возможность генерации поля флуктуаций изменяющееся во времени;
\end{itemize}
\begin{itemize}[label=--]
    \item Точность аппроксимации целевого спектра требует подбора параметров;
    \item Большое количество параметров влияющих на результат генерации;
    \item Наличие конечного периода у сгенерированных функций.
\end{itemize}
Достоинства и недостатки метода прямого моделирования Гаусса:
\begin{itemize}[label=+]
    \item Точное отражение энергетического спектра при использовании полного набора собственных чисел.
\end{itemize}
\begin{itemize}[label=--]
    \item Вычисление полного набора собственных чисел требует слишком больших вычислительных ресурсов и невозможно в практических приложениях;
    \item Использование неполного набора собственных чисел ведёт к быстрой деградации качества решения.
\end{itemize}
Достоинства и недостатки метода последовательных симуляций:
\begin{itemize}[label=+]
    \item Высокая точность аппроксимации целевого спектра в инерционном интервале;
    \item Хорошая устойчивость к топологии сетки;
    \item Возможность генерировать бесконечные непериодические пульсационные поля.
\end{itemize}
\begin{itemize}[label=--]
    \item Меньшая точность аппроксимации целевого спектра в диапазоне малых волновых чисел;
    \item Нет возможности генерации полей флуктуаций во времени;
    \item Большее время генерации ($\sim$3 раза) по сравнению со спектральным методом.
\end{itemize}

Таким образом, можно сформулировать следующие рекомендации по использованию методов генерации турбулентности.
Если необходима более точная аппроксимация целевого спектра более предпочтительным явялется метод последовательных симуляций.
В свою очередь, спектральный метод более подходит для случаев,
когда необходима быстрая генерации полей для постановки граничных условий в гидродинамических расчётах,
и при этом границы расчётной области достаточно далеко отодвинуты от зоны основного интереса.

Три представленных алгоритма были реализованы программно. Код библиотеки на языке C++ выложен в открытый доступ~\cite{mycode}.
