%% Согласно ГОСТ Р 7.0.11-2011:
%% 5.3.3 В заключении диссертации излагают итоги выполненного исследования, рекомендации, перспективы дальнейшей разработки темы.
%% 9.2.3 В заключении автореферата диссертации излагают итоги данного исследования, рекомендации и перспективы дальнейшей разработки темы.

По итогам данной диссертационной работы были реализованы и усовершенствованы подходы к генерации синтетической турбулентности. 

Первый из них основан на широкораспространённом спектральном методе. Предлагаемые усовершенствования нацелены не только на ускорение вычислительного алгоритма, но также и на лучшее удовлетворение целевым условиям, в виде лучшей аппроксимации задаваемого спектра. Переход к использованию одной гармоники Фурье позволяет существуенно сократить вычислительные затраты на генерацию одной флуктуации. Явное выражение амплитуд мод Фурье с использованием энергетического спектра позволяет проводить более точную аппроксимацию спектра, что в свою очередь ведёт к более реалистичному генерируемому полю турбулентных флуктуаций.

Метод последовательных симуляций ранее не использовался для генерации турбулентных флуктуаций, таким образом необходимо уделить особое внимание валидации генерируемых полей. Так, в сравнении с модифицированным спектральным методом мы показали, что данный предлагаемый подход является конкуретным для подобного рода задач, на примере совпадения целевого спектра, как главного критерия валидации. Последовательные симуляции позволили не только в разы сократить требуемое для генерации время, но также позволили рассматривать более подробные сетки и генерировать на них турбулентные флуктуации за счёт существенного уменьшения размерности результирующих матриц. Метод показал отличное совпадение с целевым спектром в инерционном интервале и показал более точное совпадение по сравнению со спектральным методом. 

Суммируя полученные результаты, кратко выделим преимущества и недостатки рассматриваемых методов.

Спектральный метод достоинства:
\begin{enumerate}
    \item Высокая скорость генерации одной реализации, в среднем $44 мс$, без ипользования параллельных вычислений;
    \item Устойчивость к топологии сетки, нет зависимости от соседник узлов;
    \item Возможность генерации поля флуктуаций изменяющееся во времени;
\end{enumerate}

Спектральный метод недостатки:
\begin{enumerate}
    \item Точность аппроксимации целевого спектра требует подбора параметров;
    \item Большое количество параметров влияющих на результат генерации;
\end{enumerate}

Метод последовательных симуляций достоинства:
\begin{enumerate}
    \item Высокая точность аппроксимации целевого спектра в инерционном интервале;
    \item Хорошая устойчивость к топологии сетки.
\end{enumerate}

Метод последовательных симуляций недостатки:
\begin{enumerate}
    \item Меньшая точность аппроксимации целевого спектра в диапазоне малых волновых чисел;
    \item Нет возможности генерации полей флуктуаций во времени;
    \item Большее время генерации (~3 раза) по сравнению со спектральным методом.
\end{enumerate}

На данный момент сложно выделить более подходящий метод, который был бы универсален в любой ситуации. Если необходима более точная аппроксимация целевого спектра более предпочтительным явялется метод последовательных симуляций. Меньшее число параметров позволяет сконцетрироваться на их подборе для получения лучшего результата. В свою очередь, спектральный метод будет хорошо если имеется необходимость в быстрой генерации полей и имеется достаточное расстояние для развития турбулентности из синтетической к физической. 

Отдельным пунктом можно рассмотреть зависимость от времени. Мы можем использовать оба метода для генерации граничных условий, граница является некоторой поверхностью, тем самым мы можем интерпретировать одну из осей как время и снимать вектора флуктуаций с ортгональной к этой оси плоскости. Для специфических задач, в которых граница это сложная область, в которой затруднительно провести генерацию на её поверхности с последующими смещениями для моделирования течения времени, более предпочтительным будет является спектральный метод.

В результате работы были решены поставленные задачи, предложен новый подход к генерации синтетической турбулентности. Была написана баблиотека генерации описанными в работе методами и выложена как открытый проект под лицензией MIT.

Планами на дальнейшую работу являются решение недостатков каждого из методов, возможно их объединение для совмещения сильных сторон обоих методов: точность аппроксимации целевого спектра, возможность генерации полей флуктуации во времени -- в конечном итоге проведение моделирования процессов с использованием предложенных генераторов турбулентности.
