%% Согласно ГОСТ Р 7.0.11-2011:
%% 5.3.3 В заключении диссертации излагают итоги выполненного исследования, рекомендации, перспективы дальнейшей разработки темы.
%% 9.2.3 В заключении автореферата диссертации излагают итоги данного исследования, рекомендации и перспективы дальнейшей разработки темы.

Первоначальное сравнение методов дало следующие преимущества и недостатки спектрального метода и стохастического метода друг с другом.

Скорость исполнения выше для спектрального метода, даже с учётом того факта, что используемая библиотека линейной алгебры проводит распараллеливание вычислений собственных векторов и значений для ковариационной матрицы. Но это касательно лишь единичной генерации. Если число необходимых сгенерированных полей существенно растёт, более выгоден с точки зрения затрачиваемого времени стохастической метод, так как решив систему один раз, любая последующая генерация ведёт за собой простейшее умножение некоторого сгенерированного вектора на собственные значения, что намного быстрее чем проведение алгоритма спектрального метода.

В плане требуемой памяти также более предпочтительным выбором может стать спектральный метод, так как для генерации флуктуации необходимо хранить лишь сгенерированные случайные числа, что сильно экономнее начального хранения матрицы (даже в разреженном виде) в методе стохастического моделирования, без учёта последующего требования памяти на хранение собственных векторов. Но также после проведения процедуры нахождения собственных чисел и векторов, при небольшом их взятом числе, стоимость по памяти может быть близка к спектральному методу.

Для оценки удовлетворения задаваемым условиям необходимо проводить больше тестов, но уже сейчас можно сказать, что оба метода занимают хорошее положение по точности удовлетворению задаваемым требованиям. С последующим изучением параметров влияющих на результаты можно добить ещё большего согласования с целевыми спектрами.
