Одной из центральных задач гидродинамики является изучение турбулентного течения --
движения жидкости, которое характеризуется наличием беспорядочных разномасштабных вихревых структур в потоке.
Для расчёта таких течений широко распространённой практикой является применение методов численного моделирования.
Принципиальной трудностью моделирования турбулентных течений является
наличие мелких вихревых структур, которые, тем не менее, оказывают
большое влияние на характеристики основного потока.
Подходы к численному моделированию турбулентности можно разделить на две большие группы.
В методах прямого численного моделирования (DNS) используют
приближения очень высокого пространственного разрешения, позволяющее учитывать
завихрения всех имеющихся масштабов.
Более широкая группа методов усреднённого моделирования (LES, RANS и т.д.)
основана на вычислении только крупномасштабных составляющих
потока, эффект же от мелких флуктаций описывается
за счёт добавления дополнительных слагаемых
в определяющие уравнения.

Методы первой группы предельно требовательны
к вычислительным ресурсам.
Одним из способов уменьшения вычислительных затрат
является минимизация объема пространства и промежутка времени,
на котором производится основной расчёт,
что требует постановки адекватных начальных и граничных условий для течения.
То есть возникает необходимость в умении искусственно генерировать турбулентные течения
по заранее заданным свойствам.

В методах крупномасштабного моделирования
параметры, описывающие турбулентные свойства потока, как правило
явно присутствуют в постановке задачи. Тем не менее,
многие из известных подходов так же 
имеют достаточно высокую разрешающую способность,
чтобы отражать турбулентные завихрения до некоторого предела.
Что так же приводит к необходимости задания физичного турбулентного потока
на расчётных границах.
Кроме того, синтетическая турбулентность используется в качестве добавки к полученному 
крупномасштабному решению для получения ``истинной'' картины течения.

Таким образом, задача об искусственной генерации турбулентного потока
является актуальной проблемой вычислительной гидромеханики.
Развитие и оптимизация алгоритмов генерации турбулентных флуктуаций в перспективе позволит
не только снизить расчетные и временные затраты на моделирование,
но и повысить точность предсказания интегральных характеристик потоков,
сократив разрыв между модельным экспериментом и исследуемой физической системой. 

Среди известных подходов для генерации турбулентного потока можно выделить:
использование более простых характерных течений~\cite{schluter2004large},
рециркуляция турбулентности~\cite{lund1998generation,spalart2006direct,shur2011rapid,araya2011dynamic},
генерация синтетической турбулентности~\cite{Kraichnan70,Smirnov2001,huang2010general,shur2014synthetic,adamian2011efficient,batten2004interfacing},
стохастическое моделирование,
искусственное форсирование или введение объемных источников~\cite{gritskevich2012embedded,spille2001generation},
введение генераторов вихрей~\cite{terracol2016investigation}.
Каждый из данных подходов имеет свои достоинства и недостатки, но наиболее привлекательными выглядят
подходы, основанные на статистическом описании турбулентных процессов:
подход генерации синтетической турбулентности STG (synthetic turbulence generation) и стохастическое моделирование.
По сравнению с остальными эти методы имеют такие преимущества как малая зона адаптации и возможность контроля паразитных шумов.
С физической стороны они позволяют удовлетворить наперед заданным статистическим характеристикам турбулентности.  

Фундаметнальные основы статистического подхода к описанию турбулентности были заложены в работах А.Н.~Колмогорова, А.М.~Обухова, М.Д.~Миллионщикова и др. 
Согласно этому подходу, скорость потока $U$ раскладывается на сумму средней $U_0$ и пульсационной $u$
составляющих:
$$\vec{U}\left(\vec{x}, t\right) = \vec{U_0}\left(\vec{x}, t\right)  + \vec{u}\left(\vec{x}, t\right).$$
Пульсации скорости -- есть случайная величина, которая в случае однородной изотропной
турбулентности полностью характеризуется функцией энергетического спектра $E(k)$,
которая показывает вклад в общую кинетическую энергию пульсаций колебаний с волновым числам $k$.

Широко распространённый STG-подход заключается в разложении пульсации скорости в конечный ряд по гармоникам,
амплитуды и фазы которых выбираются с помощью генератора случайных чисел с наложением дополнительных ограничений.
Стохастическое моделирование основано на восстановлении реализации случайного процесса по заданным ковариационным функциям.

Целью настоящей диссертационной работы является исследование статистичесих подходов к генерации турбулентности
по заданному энергетическому спектру в рамках однородной изотропной модели,
сравнение методов (как по вычислительной сложности, так и по качеству удовлетворения заданных статистических свойств) и 
выработка рекомендаций к их применению в прикладных задачах.

Предполагается решить следующие задачи:
\begin{enumerate}
	\item Разбор математической модели, используемой в статистических подходах к генерации турбулентности,
	\item Анализ и поиск способов улучшения существующих методик,
	\item Программирование и валидация алгоритмов на тестовых примерах,
	\item Сравнение двух обозначенных подходов с точки зрения производительности и качества решения. Выделение их достоинств и недостатков.
\end{enumerate}
