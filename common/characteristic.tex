{\actuality} Одной из актуальных задач вычислительной аэрогидромеханики в части моделирования движения флюидов является генерация турбулентных флуктуаций для входных потоков на уровне граничных условий. Не менее актуально и применение генерации турбулентных флуктуаций на поверхностях (интерфейсах) раздела или границах областей математических и вычислительных моделей, отличающихся методом моделирования.  

Применение методов генерации турбулентных флуктуаций позволяет компенсировать недоставки широко применяемого метода набегающего ламинарного потока в части прогнозирования значений интегральных параметров и минимизации вычислительных и временных затрат. Развитие и оптимизация алгоритмов генерации турбулентных флуктуаций позволит (обеспечит) в перспективе не только снизить расчетные и временные затраты на моделирование, но и повысить точность предсказания интегральных характеристик   потоков. В частности, реальный поток флюида, будь то движение смесей в трубах, или обтекание профилей, имеет турбулентный характер. Поэтому генерация турбулентных флуктуаций может сократить разрыв между модельным экспериментом и исследуемой физической системой. 

Таким образом, целью научно-исследовательской работы является разработка алгоритма для генерации турбулентных флуктуаций. 

В соответствии с целью НИР предполагается решить следующие задачи: 

\begin{enumerate}
    \item Разработка математического базиса, для наиболее четкого задания турбулентных флуктуаций, от чего зависит область адаптации турбулентности 
    \item Программирование алгоритма. Данная задача подразумевает, выбор языка программирования, разработку архитектуры программы, интеграция с существующими коммерческими и открытыми пакетами вычислительной гидродинамики.
    \item Валидация алгоритма на удовлетворение задаваемым условиям 
\end{enumerate}

Среди известных подходов для генерации турбулентного потока имеется: использование более простых характерных течений, рециркуляция турбулентности, генерация синтетической турбулентности, искусственное форсирование или введение объемных источников, введение генераторов вихрей. Каждый из данных подходов имеет свои достоинства и недостатки, но среди всех, особенно отличается подход генерации синтетической турбулентности STG (synthetic turbulence generation). Данный подход уже сейчас активно используется в решении современных комплексных задач моделирования сложных течений. По сравнению с остальными имеет такие привлекательные характеристики как: малая зона адаптации, возможность контроля паразитных шумов. С физической стороны, данный алгоритм позволяет удовлетворить наперед заданным спектрам и характеристикам турбулентности.  

В основе метода генерации синтетической турбулентности лежит представление флуктуаций компонент скоростей течения в виде конечных рядов Фурье. Большое число параметров таких как: амплитуды гармоник, волновые числа, частоты – позволит более точно удовлетворять требуемому спектру и характеристикам турбулентности, от чего зависит качество турбулентности. Помимо этого, данный метод не требует дискретизации по конечно-элементной сетке, так как флуктуации заданы в виде пространственно-временного ряда, вычисляемого в любой точке пространства и любой момент времени.  

Также помимо метода генерации синтетической турубулентности можно использовать метод стохастического Гауссова моделирования. С помощью данного метода можно также получить поле скоростей удовлетворяющее заданному энергитическому спектру. В отличии от методов генерации синтетической турбулентности, где есть возможность генерировать флуктуации в любой точке, метод стохастического моделирования требует решения сеточного уравнения, в конечном этоге решение задачи на собственные числа и векторы матрицы. Для генерации с использованием данного подхода требуется больше вычислительных ресурсов, но этап генерации более быстр. Также результирующее поле лучше удовлетворяет заданным критериям.

В конечном итоге, необходимо провести сравнение методов, как по вычислительной сложности и требованию ресурсов вычислительной машины, так и на удовлетворение поставленной задаче. Метод генерации синтетической турбулентности используется для сравнения предлагаемого стохастиеского метода, в следсвтии его большей распространённости в литературе и в современных пакетах вычислительной гидродинамики.

Проведение сравнения двух описанных ранее методов позволит наметить направление на улучшение методов, возможно их объединение, либо выборе одного более предпочтительного для дальнейшего исследования, а также валидации на примере некоторых реальных течений, что в будущем позволить существенно изменить подход к моделированию турбулентных потоков. 
