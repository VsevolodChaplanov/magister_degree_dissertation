\subsection*{Общая характеристика работы}

\newcommand{\actuality}{\underline{\textbf{Актуальность темы.}}}
\newcommand{\aim}{\underline{\textbf{Целью}}}
\newcommand{\tasks}{\underline{\textbf{задачи}}}
\newcommand{\defpositions}{\underline{\textbf{Основные положения, выносимые на~защиту:}}}
\newcommand{\novelty}{\underline{\textbf{Научная новизна:}}}
\newcommand{\influence}{\underline{\textbf{Практическая значимость}}}
\newcommand{\reliability}{\underline{\textbf{Достоверность}}}
\newcommand{\probation}{\underline{\textbf{Апробация работы.}}}
\newcommand{\contribution}{\underline{\textbf{Личный вклад.}}}
\newcommand{\publications}{\underline{\textbf{Публикации.}}}

Одной из центральных задач гидродинамики является изучение турбулентного течения --
движения жидкости, которое характеризуется наличием беспорядочных разномасштабных вихревых структур в потоке.
Для расчёта таких течений широко распространённой практикой является применение методов численного моделирования.
Принципиальной трудностью моделирования турбулентных течений является
наличие мелких вихревых структур, которые, тем не менее, оказывают
большое влияние на характеристики основного потока.
Подходы к численному моделированию турбулентности можно разделить на две большие группы.
В методах прямого численного моделирования (DNS) используют
приближения очень высокого пространственного разрешения, позволяющее учитывать
завихрения всех имеющихся масштабов.
Более широкая группа методов усреднённого моделирования (LES, RANS и т.д.)
основана на вычислении только крупномасштабных составляющих
потока, эффект же от мелких флуктаций описывается
за счёт добавления дополнительных слагаемых
в определяющие уравнения.

Методы первой группы предельно требовательны
к вычислительным ресурсам.
Одним из способов уменьшения вычислительных затрат
является минимизация объема пространства и промежутка времени,
на котором производится основной расчёт,
что требует постановки адекватных начальных и граничных условий для течения.
То есть возникает необходимость в умении искусственно генерировать турбулентные течения
по заранее заданным свойствам.

В методах крупномасштабного моделирования
параметры, описывающие турбулентные свойства потока, как правило
явно присутствуют в постановке задачи. Тем не менее,
многие из известных подходов так же 
имеют достаточно высокую разрешающую способность,
чтобы отражать турбулентные завихрения до некоторого предела.
Что так же приводит к необходимости задания физичного турбулентного потока
на расчётных границах.
Кроме того, синтетическая турбулентность используется в качестве добавки к полученному 
крупномасштабному решению для получения ``истинной'' картины течения.

Таким образом, задача об искусственной генерации турбулентного потока
является актуальной проблемой вычислительной гидромеханики.
Развитие и оптимизация алгоритмов генерации турбулентных флуктуаций в перспективе позволит
не только снизить расчетные и временные затраты на моделирование,
но и повысить точность предсказания интегральных характеристик потоков,
сократив разрыв между модельным экспериментом и исследуемой физической системой. 

Среди известных подходов для генерации турбулентного потока можно выделить:
использование более простых характерных течений~\cite{schluter2004large},
рециркуляция турбулентности~\cite{lund1998generation,spalart2006direct,shur2011rapid,araya2011dynamic},
генерация синтетической турбулентности~\cite{Kraichnan70,Smirnov2001,huang2010general,shur2014synthetic,adamian2011efficient,batten2004interfacing},
стохастическое моделирование,
искусственное форсирование или введение объемных источников~\cite{gritskevich2012embedded,spille2001generation}, введение генераторов вихрей~\cite{terracol2016investigation}.
Каждый из данных подходов имеет свои достоинства и недостатки, но наиболее привлекательными выглядят
подходы, основанные на статистическом описании турбулентных процессов:
подход генерации синтетической турбулентности STG (synthetic turbulence generation) и стохастическое моделирование.
По сравнению с остальными эти методы имеют такие преимущества как малая зона адаптации и возможность контроля паразитных шумов.
С физической стороны они позволяют удовлетворить наперед заданным статистическим характеристикам турбулентности.  

Фундаметнальные основы статистического подхода к описанию турбулентности были заложены в работах Колмогорова, Обухова, Миллионщикова и др. 
Согласно этому подходу, скорость потока $U$ раскладывается на сумму средней и пульсационной 
составляющих:
$$\vec{U}\left(\vec{x}, t\right) = \vec{U_0}\left(\vec{x}, t\right)  + \vec{u}\left(\vec{x}, t\right).$$
Пульсации скорости -- есть случайная величина, которая в случае однородной изотропной
турбулентности полностью характеризуется функцией энергетического спектра $E(k)$,
которая показывает вклад в общую кинетическую энергию пульсаций колебаний с волновым числам $k$.

Широко распространённый STG-подход заключается в разложении пульсации скорости в конечный ряд по гармоникам,
амплитуды и фазы которых выбираются с помощью генератора случайных чисел с наложением дополнительных ограничений.
Стохастическое моделирование основано на восстановлении реализации случайного процесса по заданным ковариационным функциям.

Целью настоящей диссертационной работы является исследование статистичесих подходов к генерации турбулентности
по заданному энергетическому спектру в рамках однородной изотропной модели,
сравнение методов (как по вычислительной сложности, так и по качеству удовлетворения заданных статистических свойств) и 
выработка рекомендаций к их применению в прикладных задачах.

Предполагается решить следующие задачи:
\begin{enumerate}
	\item Разбор математической модели, используемой в статистических подходах к генерации турбулентности,
	\item Анализ и поиск способов улучшения существующих методик,
	\item Программирование и валидация алгоритмов на тестовых примерах,
	\item Сравнение двух обозначенных подходов с точки зрения производительности и качества решения. Выделение их достоинств и недостатков.
\end{enumerate}
 % Характеристика работы по структуре во введении и в автореферате не отличается (ГОСТ Р 7.0.11, пункты 5.3.1 и 9.2.1), потому её загружаем из одного и того же внешнего файла, предварительно задав форму выделения некоторым параметрам

%Диссертационная работа была выполнена при поддержке грантов ...

%\underline{\textbf{Объем и структура работы.}} Диссертация состоит из~введения, четырех глав, заключения и~приложения. Полный объем диссертации \textbf{ХХХ}~страниц текста с~\textbf{ХХ}~рисунками и~5~таблицами. Список литературы содержит \textbf{ХХX}~наименование.

%\newpage
\subsection*{Содержание работы}
Во \underline{\textbf{введении}} обосновывается актуальность исследований, проводимых в рамках данной диссертационной работы, приводится обзор научной литературы по изучаемой проблеме, формулируется цель, ставятся задачи работы, сформулированы научная новизна и практическая значимость представляемой работы.

\underline{\textbf{Первая глава}} посвящена ...

 картинку можно добавить так:
\begin{figure}[ht] 
  \center
  \includegraphics [scale=0.27] {latex}
  \caption{Подпись к картинке.} 
  \label{img:latex}
\end{figure}

Формулы в строку без номера добавляются так:
\[ 
  \lambda_{T_s} = K_x\frac{d{x}}{d{T_s}}, \qquad
  \lambda_{q_s} = K_x\frac{d{x}}{d{q_s}},
\]

\underline{\textbf{Вторая глава}} посвящена исследованию 

\underline{\textbf{Третья глава}} посвящена исследованию 

В \underline{\textbf{четвертой главе}} приведено описание 

В \underline{\textbf{заключении}} приведены основные результаты работы, которые заключаются в следующем:
%% Согласно ГОСТ Р 7.0.11-2011:
%% 5.3.3 В заключении диссертации излагают итоги выполненного исследования, рекомендации, перспективы дальнейшей разработки темы.
%% 9.2.3 В заключении автореферата диссертации излагают итоги данного исследования, рекомендации и перспективы дальнейшей разработки темы.

По итогам данной диссертационной работы были реализованы и усовершенствованы подходы к генерации синтетической турбулентности. 

Первый из них основан на широкораспространённом спектральном методе. Предлагаемые усовершенствования нацелены не только на ускорение вычислительного алгоритма, но также и на лучшее удовлетворение целевым условиям, в виде лучшей аппроксимации задаваемого спектра. Переход к использованию одной гармоники Фурье позволяет существуенно сократить вычислительные затраты на генерацию одной флуктуации. Явное выражение амплитуд мод Фурье с использованием энергетического спектра позволяет проводить более точную аппроксимацию спектра, что в свою очередь ведёт к более реалистичному генерируемому полю турбулентных флуктуаций.

Метод последовательных симуляций ранее не использовался для генерации турбулентных флуктуаций, таким образом необходимо уделить особое внимание валидации генерируемых полей. Так, в сравнении с модифицированным спектральным методом мы показали, что данный предлагаемый подход является конкуретным для подобного рода задач, на примере совпадения целевого спектра, как главного критерия валидации. Последовательные симуляции позволили не только в разы сократить требуемое для генерации время, но также позволили рассматривать более подробные сетки и генерировать на них турбулентные флуктуации за счёт существенного уменьшения размерности результирующих матриц. Метод показал отличное совпадение с целевым спектром в инерционном интервале и показал более точное совпадение по сравнению со спектральным методом. 

Суммируя полученные результаты, кратко выделим преимущества и недостатки рассматриваемых методов.

Спектральный метод достоинства:
\begin{enumerate}
    \item Высокая скорость генерации одной реализации, в среднем $44 мс$, без ипользования параллельных вычислений;
    \item Устойчивость к топологии сетки, нет зависимости от соседник узлов;
    \item Возможность генерации поля флуктуаций изменяющееся во времени;
\end{enumerate}

Спектральный метод недостатки:
\begin{enumerate}
    \item Точность аппроксимации целевого спектра требует подбора параметров;
    \item Большое количество параметров влияющих на результат генерации;
\end{enumerate}

Метод последовательных симуляций достоинства:
\begin{enumerate}
    \item Высокая точность аппроксимации целевого спектра в инерционном интервале;
    \item Хорошая устойчивость к топологии сетки.
\end{enumerate}

Метод последовательных симуляций недостатки:
\begin{enumerate}
    \item Меньшая точность аппроксимации целевого спектра в диапазоне малых волновых чисел;
    \item Нет возможности генерации полей флуктуаций во времени;
    \item Большее время генерации (~3 раза) по сравнению со спектральным методом.
\end{enumerate}

На данный момент сложно выделить более подходящий метод, который был бы универсален в любой ситуации. Если необходима более точная аппроксимация целевого спектра более предпочтительным явялется метод последовательных симуляций. Меньшее число параметров позволяет сконцетрироваться на их подборе для получения лучшего результата. В свою очередь, спектральный метод будет хорошо если имеется необходимость в быстрой генерации полей и имеется достаточное расстояние для развития турбулентности из синтетической к физической. 

Отдельным пунктом можно рассмотреть зависимость от времени. Мы можем использовать оба метода для генерации граничных условий, граница является некоторой поверхностью, тем самым мы можем интерпретировать одну из осей как время и снимать вектора флуктуаций с ортгональной к этой оси плоскости. Для специфических задач, в которых граница это сложная область, в которой затруднительно провести генерацию на её поверхности с последующими смещениями для моделирования течения времени, более предпочтительным будет является спектральный метод.

В результате работы были решены поставленные задачи, предложен новый подход к генерации синтетической турбулентности. Была написана баблиотека генерации описанными в работе методами и выложена как открытый проект под лицензией MIT.

Планами на дальнейшую работу являются решение недостатков каждого из методов, возможно их объединение для совмещения сильных сторон обоих методов: точность аппроксимации целевого спектра, возможность генерации полей флуктуации во времени -- в конечном итоге проведение моделирования процессов с использованием предложенных генераторов турбулентности.



%\newpage
При использовании пакета \verb!biblatex! список публикаций автора по теме
диссертации формируется в разделе <<\publications>>\ файла
\verb!../common/characteristic.tex!  при помощи команды \verb!\nocite! 

\ifthenelse{\equal{\thebibliosel}{0}}{% Встроенная реализация с загрузкой файла через движок bibtex8
  \renewcommand{\refname}{\large \authorbibtitle}
  \nocite{*}
  \insertbiblioauthor                          % Подключаем Bib-базы
  %\insertbiblioother   % !!! bibtex не умеет работать с несколькими библиографиями !!!
}{% Реализация пакетом biblatex через движок biber
  \insertbiblioauthor                          % Подключаем Bib-базы
  \insertbiblioother
}

